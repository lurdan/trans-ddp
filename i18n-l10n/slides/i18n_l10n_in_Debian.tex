\documentclass{beamer}


\mode<presentation>
{
  \usetheme{Boadilla}
  \setbeamercovered{transparent}
}

\usepackage[utf8x]{inputenc}
\usepackage{colortbl}
\usepackage[english]{babel}

\usepackage{times}

\setbeamercovered{dynamic}


\title[State of the art of i18n/l10n in Debian] 
{State of the art of i18n/l10n in Debian}

\author[bubulle,jfs]
{Christian ~Perrier, Javier ~Fernández-Sanguino Peña}

\date[Debconf 6] 
{7th Debian Conference, Oaxtepec, Mexico}

\pgfdeclareimage[height=2cm]{debian-logo}{debian-swirl}
\logo{\pgfuseimage{debian-logo}}



\AtBeginSection[]
{
  \begin{frame}<beamer>
    \frametitle{}
    \tableofcontents[sectionstyle=show/hide,subsectionstyle=show/show/hide]
  \end{frame}
}

%\AtBeginSubSection[]
%{
%  \begin{frame}<beamer>
%    \frametitle{Talk structure}
%    \tableofcontents[sectionstyle=show/hide,subsectionstyle=show/show/hide]
%  \end{frame}
%}


\beamerdefaultoverlayspecification{}


\pgfdeclareimage[width=11cm]{potato}{potato}
\pgfdeclareimage[width=11cm]{woody}{woody}
\pgfdeclareimage[width=11cm]{sarge}{sarge}
\pgfdeclareimage[width=11cm]{etch_beta}{etch_beta}
\pgfdeclareimage[width=11cm]{etch}{etch}
\pgfdeclareimage[width=6cm]{spongebob}{spongebob}

% \logo{\vbox{\hbox to 1cm{\hfil\pgfuseimage{spongebob}}}}

\begin{document}

\frame{\titlepage}

\begin{frame}<beamer>
    \frametitle{Talk and round table Structure}
    \tableofcontents[subsectionstyle=hide]
\end{frame}

% begin Christian
\section{Internationalisation...again}


\begin{frame}
  \frametitle{English still sucks}
	\begin{block}
		{De facto communication language?}
	\end{block}
	\begin{block}
		{Covers 5\% to 15\% of the world population}
	\end{block}
	\begin{block}
		{OS only in English is not a universal OS}
	\end{block}
	\begin{block}
		{Even not all geeks speak English}
	\end{block}
\end{frame}

\begin{frame}
  \frametitle{Towards World Domination}
  \animate<2-4>
	\only<1>{
		\begin{block}<1>
		{D-I Potato: 1  language}
		\end{block}
	}
	\only<2>{
	\begin{block}<2>
		{D-I Woody: 16 languages}
	\end{block}
	}
	\only<3>{
	\begin{block}<3>
		{D-I Sarge: 42 languages}
	\end{block}
	}
	\only<4>{
	\begin{block}<4>
		{D-I Etch beta2: 53 languages}
	\end{block}
	}
	\only<5>{
	\begin{block}<5>
		{D-I Etch: 63 languages}	
	\end{block}
	}
	\only<6>{
		\begin{block}<6>
		{D-I Potato: 1  language}
		\end{block}
	}
	\only<7>{
	\begin{block}<7>
		{D-I Woody: 16 languages}
	\end{block}
	}
	\only<8>{
	\begin{block}<8>
		{D-I Sarge: 42 languages}
	\end{block}
	}
	\only<9>{
	\begin{block}<9>
		{D-I Etch beta2: 53 languages}
	\end{block}
	}
	\only<10>{
	\begin{block}<10>
		{D-I Etch: 63 languages}	
	\end{block}
	}
	\only<11>{
	\begin{block}<11>
		{Champagne}	
	\end{block}
	}
  \pgfuseimage{potato}<1>
  \pgfuseimage{woody}<2>
  \pgfuseimage{sarge}<3>
  \pgfuseimage{etch_beta}<4>
  \pgfuseimage{etch}<5>
  \pgfuseimage{potato}<6>
  \pgfuseimage{woody}<7>
  \pgfuseimage{sarge}<8>
  \pgfuseimage{etch_beta}<9>
  \pgfuseimage{etch}<10>
  \pgfuseimage{spongebob}<11>

\end{frame}

\begin{frame}
  \frametitle{Maintainers/Translators wars are over}
	\begin{block}
		{Up to 2000-2002}
		i18n/l10n as the last wheel
	\end{block}
	\begin{block}
		{FLOSS coverage increasing}
		i18n/l10n closer to the heart of development issues and teams
	\end{block}
\end{frame}

\begin{frame}
  \frametitle{Please, no more skirmishes}
	\begin{block}
		{Translations are good for your code / packages}
		\begin{itemize}
		\item Add them as soon as possible, do not wait for a new release.
		\item Consider translators valuable contributors and treat them as such.
		\end{itemize}
	\end{block}
	\begin{block}
		{Some helpful advice}
		\begin{itemize}
		\item Translations help reviewing your language use and consistency.
		\item Do string freezes.
		\item Ask for translations if you need them done or updated.
		\end{itemize}
	\end{block}
\end{frame}

% end Christian

\section{i18n/l10n projects}

% begin Javier
\subsection{Web site}

\begin{frame}
  \frametitle{Web site translation}
	\begin{block}
		{Translated content}
		\begin{itemize}
		\item Available through Content Negotiation
		\item Managed with:
		\begin{itemize}
		\item Translation headers
		\item Gettext
		\end{itemize}
		\end{itemize}
	\end{block}
	\begin{block}
		{Status}
		\begin{itemize}
		\item Translated to 33 languages
		\item \href{http://www.debian.org/devel/website/stats}{Stats}: 6 languages over 50\% pages translated (7 if counting size \%)
		\end{itemize}
	\end{block}
\end{frame}

\subsection{Documentation}

\begin{frame}
  \frametitle{Documentation translation}
	\begin{block}
		{Documentation availability}
		\begin{itemize}
		\item DDP CVS access for translators
		\item Some documents outside DDP
		\item Starting to use doc-check / poxml
		\end{itemize}
	\end{block}
	\begin{block}
		{Manpages}
		\begin{itemize}
		\item Internal project to use po4a
		\item Some handled outside upstream (manpages-XX)
		\item Big mess.
		\end{itemize}
	\end{block}
\end{frame}

\begin{frame}
  \frametitle{Documentation translation status}
	\begin{block}
		{Overall Status}
		\begin{itemize}
		\item DDP:
% In the DDP directory:
% (for i in *; do [ -d $i ] && find $i  -name "*.??.sgml" \
% | perl -pe 's/^.*\.(.*?)\.sgml/$1/' |sort -u; [ -d $i ] && ls -d $i/?? | \
% perl -pe 's/^.*\/(??)/$1/' ; done ) | sort | uniq -c |less
		\begin{itemize}
		\item Release Notes: 11 languages
		\item More than 20 different language teams
		\item Only some teams translated over 10 docs (of 40)
		\end{itemize}
		\item D-i Manual: 15 languages
		\item Manpages: 
		\item Better than other FLOSS projects
		\end{itemize}
	\end{block}
\end{frame}

% end Javier

\subsection{Debian packages}

% begin Javier
\subsubsection{Packages descriptions}

\begin{frame}
  \frametitle{Packages descriptions (DDTP)}
	\begin{block}
		{Demanded by our users}
		\begin{itemize}
		\item {\em What should I install?}
		\item {\em What is this software doing in my system?}
		\end{itemize}
	\end{block}
	\begin{block}
		{Available at http://ddtp.debian.org/}
	\end{block}
	\begin{block}
		{Issues}
		\begin{itemize}
		\item Project was abandoned (FIXED)
		\item Needed APT changes (DONE in 0.6)
% Why does Ubuntu require login for *exporting* translations???, jfs
		\item \href{https://launchpad.net/products/ddtp-ubuntu/+translations}{Ubuntu translations} (?)
		\item Ftpmasters have to introduce it in the archive.
		\end{itemize}
	\end{block}
\end{frame}


% end Javier

% begin Christian

\subsubsection{Debconf templates}

\begin{frame}
  \frametitle{Debconf templates}
	\begin{block}
		{po-debconf: the simple and efficient tool}
	\end{block}
	\begin{block}
		{Writing style and overall consistency}
		\begin{itemize}
		\item
			\href{http://www.debian.org/doc/developers-reference/ch-best-pkging-practices.html\#s-bpp-config-mgmt}{section 6.5 of the Developer's Reference}
		\item
			Listen to lintian
		\end{itemize}
	\end{block}
	\begin{block}
		{Communicate with translators}
		\begin{itemize}
		\item
			podebconf-report-po
		\end{itemize}
	\end{block}
	\begin{block}
		{Provide up-to-date POT files}
		\begin{itemize}
		\item
			debconf-updatepo in clean target (?)
		\end{itemize}
	\end{block}
\end{frame}

\subsubsection{Native Debian packages}

\begin{frame}
  \frametitle{Native Debian packages - TBD}
	\begin{block}
		{Programs translation}
	\end{block}
	\begin{block}
		{Documentation translation}
	\end{block}
	\begin{block}
		{Work on REAL native Debian packages}
	\end{block}
\end{frame}
% end Christian



\section{i18n/l10n infrastructure}

% begin Christian
\subsection{PO translation statistics}

\begin{frame}
  \frametitle{PO-debconf, PO statistics web pages}
	\begin{block}
		{\href{http://www.debian.org/intl/l10n/po-debconf/fr}{Early ``framework'' for localisation}}
	\end{block}
	\begin{block}
		{Statistics and access to material}
	\end{block}
	\begin{block}
		{Uses popcon stats}
	\end{block}
	\begin{block}
		{Weaknesses?}
		\begin{itemize}
		\item
			Run under developers accounts
		\item
			Only reflect uploaded packages
		\item
			No status for testing
		\end{itemize}
	\end{block}
\end{frame}
	
% begin Christian
\subsection{Installer translation statistics}

\begin{frame}
  \frametitle{Installer translation statistics}
	\begin{block}
		{\href{http://people.debian.org/~seppy/d-i/translation-status.html}{All in one page}}
	\end{block}
	\begin{block}
		{Direct access to material, grabbed from development trees}
	\end{block}
  \begin{columns}
    \begin{column}{4.5cm}
	\begin{block}
		{Weaknesses}
		\begin{itemize}
		\item
			Depend on seppy
		\item
			Sensitive to upstream changes
		\item
			Does not scale well
		\item
			Need a coordinator
		\end{itemize}
	\end{block}
    \end{column}
    \begin{column}{4.5cm}
	\begin{block}
		{Strengths}
		\begin{itemize}
		\item
			Big reactivity
		\item
			Enforced ``localisation helpers''
		\end{itemize}
	\end{block}
    \end{column}
  \end{columns}
\end{frame}

% begin Christian
\subsection{Use of the BTS for l10n work}

\begin{frame}
  \frametitle{Use of the BTS for l10n work}
	\begin{block}
		{Bug reports}
		\begin{itemize}
		\item
			Severity: wishlist
		\item
			Tags: l10n, patch
		\item
			Subject: $[$l10n: code$]$ (type) translation to (language)
		\item
			Material attached to the bug report, optionally compressed
		\end{itemize}
	\end{block}
	\begin{block}
		{Maintainer duties}
		\begin{itemize}
		\item
		{Be responsive to bug reports}
		\item
		{Apply ASAP}
		\item
		{Do translation updates during freezes}
		\end{itemize}
	\end{block}
\end{frame}

% end Christian

% begin Javier
% I don't think this provides much useful information for
% people attending the talk
%\subsection{Web site translation statistics}
%
%\begin{frame}
%  \frametitle{Web site translation statistics - TBD}
%	\begin{block}
%		{Available at \href{http://www.debian.org/devel/website/stats}{Translation statistics page}}
%	\end{block}
%\end{frame}
%
% end Javier

% begin Javier
\subsection{Translation robots}

\begin{frame}
% Maybe add a nice graphics here
  \frametitle{Translation robots: following l10n processes}
	\begin{block}
		{General translation processes}
		\begin{itemize}
		\item
			{TAF->ITT->RFR->(RFR2->)LCFC->BTS->FIX->DONE}
		\end{itemize}
	\end{block}
	\begin{block}
		{Use of mailing lists and pseudo-URLs in subjects}
		\begin{itemize}
		\item
			{\texttt{[TAF] po://dpkg/es.po XfYu}}
		\item
			{\texttt{[ITT] po://dpkg/es.po}}
		\item
			{\texttt{[RFR] po://dpkg/es.po}}
		\item
			{\texttt{[LCFC] po://dpkg/es.po}}
		\item
			{\texttt{[BTS] po://dpkg/es.po \#400000}}
		\end{itemize}
	\end{block}
	\begin{block}
		{Tracked by robots \href{http://http://www.debian.org.es/cgi-bin/l10n.cgi?team=es}{to produce status pages}}
		Currently used by 9 teams (out of 21)
% http://www.google.com/search?ie=UTF8&q=RFR+site%3Alists.debian.org
% Dutch, Turkish, Spanish, French, Romanian, Catalan, German, 
% Arabic, Portuguese
	\end{block}
\end{frame}

% end Javier

\section{i18n/l10n tools in Debian}

% begin Christian
%\subsection{Generic tools}

%\begin{frame}
%  \frametitle{Gettext as translation common ground}
%	\begin{block}
%		{Gettext in software}
%		\begin{itemize}
%		\item
%			De facto standard for i18n of Free Software
%		\item
%			\texttt{xgettext} to extract strings from source code
%		\end{itemize}
%	\end{block}
%	\begin{block}
%		{Gettext for translators}
%		\begin{itemize}
%		\item
%			Begin from POT files
%		\item
%			Maintain PO files
%		\end{itemize}
%	\end{block}
%	\begin{block}
%		{Gettext for both}
%		\begin{itemize}
%		\item
%			\texttt{aptitude install gettext}
%		\item
%			\texttt{dpkg -L gettext | grep "/bin/msg"}
%		\end{itemize}
%	\end{block}
%\end{frame}

%\begin{frame}
%  \frametitle{Working on gettext files}
%	\begin{block}
%		{Standard editors with gettext handling}
%		\begin{itemize}
%		\item
%			vim gettext mode
%		\item
%			Emacs PO-mode
%		\item
%			Limited use because of limited dedicated features
%		\end{itemize}
%	\end{block}
%	\begin{block}
%		{Dedicated tools}
%		\begin{itemize}
%		\item
%			Easy view of original, translated strings
%		\item
%			Easy access to comments
%		\item
%			Spellchecking and error checking (variable matching)
%		\item
%			Translation memory
%		\item
%			KBabel, POedit, gtranslator...
%		\end{itemize}
%	\end{block}
%\end{frame}
% end Christian

% begin Christian
\subsection{Po4a}

\begin{frame}
  \frametitle{PO4a: PO for Anything}
	\begin{block}
		{Bring gettext translations to new areas}
		\begin{itemize}
		\item
			Man pages
		\item
			Documentation
		\item
			...
		\end{itemize}
	\end{block}
	\begin{block}
		{PO4a in Debian}
		\begin{itemize}
		\item
			Man pages for native packages
		\item
			Experimental status pages for manpages
		\item
			Documentation currently prefers POXML
		\end{itemize}
	\end{block}
	\begin{block}
		{Use it in your packages}
		\begin{itemize}
		\item
			Read the best practices
		\item
			{Ask for help in \texttt{debian-i18n}}
		\item
			Profit!
		\end{itemize}
	\end{block}
\end{frame}

% begin Christian
\subsection{Po-debconf}

\begin{frame}
  \frametitle{Po-debconf tools}
	\begin{block}
		{debconf-updatepo}
		\begin{itemize}
		\item
			{Update files in \texttt{debian/po}}
		\item
			Should be run before shipping the package (clean target?)
		\end{itemize}
	\end{block}
	\begin{block}
		{po2debconf}
		\begin{itemize}
		\item
			Rebuild templates from \texttt{debian/templates} and PO files
		\item
			Integrated in \texttt{dh\_installdebconf}
		\end{itemize}
	\end{block}
	\begin{block}
		{podebconf-report-po}
		\begin{itemize}
		\item
			Warn translators for needed updates
		\end{itemize}
	\end{block}
\end{frame}

% end Christian

% begin Javier
\subsection{Doc-check}

\begin{frame}
  \frametitle{Freshness of documentation l10n}
	\begin{block}
		{doc-check}
		\begin{itemize}
		\item Script to track status of translations
		\item Uses {\em translation headers}
		\item Useful when it is not possible (or translators do not want to) use poxml / po4a
		\end{itemize}
	\end{block}
	\begin{block}
		{Status / Issues}
		\begin{itemize}
		\item Used only by 4 documents
		\item Different scripts, need to be merged
		\item Different header formats
		\end{itemize}
	\end{block}
\end{frame}

\subsection{Localization-config / tasksel}

\begin{frame}
  \frametitle{Localization-config / tasksel}
	\begin{block}
		{Translation of installed systems}
		\begin{itemize}
		\item Get the system as fully localized as possible
		\item As few manual actions as possible for users 
		\item Focus on desktop systems
		\end{itemize}
	\end{block}
	\begin{block}
		{Localization-config}
		\begin{itemize}
		\item Localize gdm, KDE, dictionaries, X keyboard selection, etc.
		\item Previously used by d-i (reintegration needed)
		\end{itemize}
	\end{block}
	\begin{block}
		{Tasksel's language tasks}
		\begin{itemize}
		\item
			{\texttt{language task}}, depends only on standard packages
		\item
			{\texttt{language-desktop task}}, l10n for desktop environments and user apps
		\item
			{Issues with task maintenance}
		\end{itemize}
	\end{block}
\end{frame}
% end Javier

\section{Relationship with other projects}

% begin Javier
\subsection{Translation packaging}

\begin{frame}
  \frametitle{Translation packaging}
	\begin{block}
		{Packages -XX}
		Maintainers sometimes package translations in indepent packages
	\end{block}
	\begin{block}
		{The Good}
		Users will install them if they need them, but not all users have to.
	\end{block}
	\begin{block}
		{The Bad}
		Sometimes independ from upstream code (i.e. manpages)
	\end{block}
	\begin{block}
		{The Ugly}
		\begin{itemize}
		\item Users think they are reading current copy
		\item Translations get out of date and we do not notice
		\item Dependency problems when translations are tied to program (thunderbird, mozilla...), breaks transition to testing
		\end{itemize}
	\end{block}
\end{frame}
% end Javier

% begin Christian
\subsection{Interaction with upstream l10n}

\begin{frame}
  \frametitle{Handling updates of upstream l10n}
	\begin{block}
		{In general: ask translators to work upstream}
	\end{block}
	\begin{block}
		{Depends on Debian-upstream interactions}
		\begin{itemize}
		\item
			Forward l10n updates to upstream
		\item
			Mark pending when upstream commits
		\item
			Close with new upstream release
		\end{itemize}
	\end{block}
	\begin{block}
		{During Debian freezes}
		\begin{itemize}
		\item
			Bring l10n updates from upstream
		\item
			Call for updates to translators
		\item
			Use the BTS
		\end{itemize}
	\end{block}
\end{frame}

%\begin{frame}
%  \frametitle{Handling errors in upstream strings}
%	\begin{block}
%		{During regular development}
%		\begin{itemize}
%		\item
%			Forward bug reports to upstream
%		\item
%			Wait for upstream to fix errors
%		\end{itemize}
%	\end{block}
%	\begin{block}
%		{During Debian freezes}
%		\begin{itemize}
%		\item
%			Forward bug reports to upstream
%		\item
%			Commit fixes to the Debian package, but:
%			\begin{itemize}
%			\item
%				Ask for fixes to Debian l10n team
%			\item
%				Commit then as patches
%			\item
%				Forward them to upstream
%			\end{itemize}
%		\end{itemize}
%	\end{block}
%\end{frame}
% end Christian

\section{Summary of Debconf BOFs}

%\begin{frame}
%     --> focus on "could be the starter for 
%         general i18n infrastructure" ?
%  \frametitle{Reviving the DDTP}
%	\begin{block}
%		{Can now be used again}
%	\end{block}
%	\begin{block}
%		{Restarting the work to settle ideas for a new i18n infrastructure}
%	\end{block}
%	\begin{block}
%		{Translators will want their work be used}
%	\end{block}
%\end{frame}

\begin{frame}
  \frametitle{Summary of i18n infrastructure BOFs}
	\begin{block}
		{We need the i18n infrastructure}
	\end{block}
	\begin{block}
		{We want it as modular-designed as possible}
	\begin{itemize}
	\item
		{It should be designed well from the beginning}
	\end{itemize}
	\end{block}
	\begin{block}
		{Two opportunities:}
	\begin{itemize}
	\item
		{Google Summer of Code}
	\item
		{Extremadura sessions: \tiny{\texttt{http://wiki.debian.org/WorkSessionExtremadura2006i18n}}}
	\end{itemize}
	\end{block}
\end{frame}


\begin{frame}
  \frametitle{Achieved specification analysis}
	\begin{block}
		{Targets of the i18n infrastructure:}
	\begin{itemize}
	\item
		{Translators}
	\item
		{Reviewers}
	\item
		{Team coordinators}
	\item
		{Maintainers}
	\item
		{Visitors}
	\item
		{Administrators (of the system)}
	\end{itemize}
	\end{block}
	\begin{block}
		{Most needs of the targets are roughly known}
	\begin{itemize}
	\item
		{Improve analysis of translator needs}
	\item
		{Improve analysis of maintainers needs}
	\item
		{Discussion in \texttt{debian-i18n@lists.debian.org}}
	\end{itemize}
	\end{block}
\end{frame}

\begin{frame}
  \frametitle{A way to go: collaboration with Wordforge}
	\begin{block}
		{Wordforge: build an opened and free framework for l10n in Free Software}
	\end{block}
	\begin{block}
		{Commited to promote open standards}
	\begin{itemize}
	\item
		{XLIFF, PO}
	\item
		{Glossaries, translation memories...}
	\item
		{Open protocols to communicate}
	\item
		{Use of computer-assisted localization technology}
	\end{itemize}
	\end{block}
	\begin{block}
		{Not going against Rosetta/Launchpad}
	\begin{itemize}
	\item
		{WordForge and Rosetta able to communicate}
	\end{itemize}
	\end{block}
	\begin{block}
		{Modularity is the key}
	\end{block}
\end{frame}


\begin{frame}
  \frametitle{Contacts}
	\begin{block}
		{Everything about us, ask Google}
		\begin{itemize}
		\item
			{Christian: bubulle@perrier.eu.org}
		\item
			{Javier: jfs@computer.org}
		\end{itemize}
	\end{block}
	\begin{block}
		{Everything about i18n}
		\begin{itemize}
		\item
			{\texttt{debian-i18n@lists.debian.org}}
		\item
			{\texttt{\#debian-i18n @ oftc}}
		\end{itemize}
	\end{block}
\end{frame}

\end{document}


