\documentclass{beamer}


\mode<presentation>
{
  \usetheme{Warsaw}
  \setbeamercovered{transparent}
}

\usepackage[utf8x]{inputenc}
\usepackage{colortbl}
\usepackage[english]{babel}

\usepackage{times}

\setbeamercovered{dynamic}


\title[State of the art of i18n/l10n in Debian] 
{State of the art of i18n/l10n in Debian}

\author[jfs, bubulle] 
{Javier ~Fernàndez-Sanguino Peña, Christian ~Perrier}

\date[Debconf 6] 
{6th Debian Conference, Oaxtepec, Mexico}

\pgfdeclareimage[height=2cm]{debian-logo}{debian-swirl}
\logo{\pgfuseimage{debian-logo}}



\AtBeginSection[]
{
  \begin{frame}<beamer>
    \frametitle{Talk structure}
    \tableofcontents[sectionstyle=show/hide,subsectionstyle=show/show/hide]
  \end{frame}
}

%\AtBeginSubSection[]
%{
%  \begin{frame}<beamer>
%    \frametitle{Talk structure}
%    \tableofcontents[sectionstyle=show/hide,subsectionstyle=show/show/hide]
%  \end{frame}
%}


\beamerdefaultoverlayspecification{}



\begin{document}

\frame{\titlepage}

\begin{frame}<beamer>
    \frametitle{Talk and round table Structure}
    \tableofcontents[subsectionstyle=hide]
\end{frame}

% begin Christian
\section{Internationalisation...again}


\begin{frame}
  \frametitle{English....or Esperanto?}
	\begin{block}<+->
		{English as de facto communication language}
	\end{block}
	\begin{block}<+->
		{Covers 5\% to 15\% of the world population}
	\end{block}
	\begin{block}<+->
		{OS only in English is not a universal OS}
	\end{block}
	\begin{block}<+->
		{Even not all geeks speak English}
	\end{block}
\end{frame}

\begin{frame}
  \frametitle{Maintainers/Translators wars are over}
	\begin{block}
		{Up to 2000-2002}
		i18n/l10n as the last wheel
	\end{block}
	\begin{block}
		{FLOSS coverage increasing}
		i18n/l10n closer to the heart of development issues and teams
	\end{block}
\end{frame}

\begin{frame}
  \animate<2-5>
  \frametitle{Towards World Domination}
  \pgfdeclareimage[width=10cm]{potato}{potato}
  \pgfuseimage{potato}<1>
  \pgfdeclareimage[width=10cm]{woody}{woody}
  \pgfuseimage{woody}<2>
  \pgfdeclareimage[width=10cm]{sarge}{sarge}
  \pgfuseimage{sarge}<3>
  \pgfdeclareimage[width=10cm]{etch_beta}{etch_beta}
  \pgfuseimage{etch_beta}<4>
  \pgfdeclareimage[width=10cm]{etch}{etch}
  \pgfuseimage{etch}<4>
\end{frame}

% end Christian

\section{i18n/l10n projects}

% begin Javier
\subsection{Web site translations}

\begin{frame}
  \frametitle{Javier}
\end{frame}

\begin{frame}
  \frametitle{Javier}
\end{frame}

\begin{frame}
  \frametitle{Javier}
\end{frame}
% end Javier

\subsection{Translation of Debian tools}

% begin Christian
\subsubsection{Debconf templates}

\begin{frame}
  \frametitle{Killer application: po-debconf}
	\begin{block}
		{Simple and efficient}
	\end{block}
	\begin{block}
		{Should become mandatory fo user interaction}
	\end{block}
\end{frame}

\begin{frame}
  \frametitle{po-debconf file layout}
	\begin{block}{Translatable material}
		\begin{itemize}
		\item
			\texttt{debian/*templates} files
		\end{itemize}
	\end{block}
	\begin{block}{Translation material}
		\begin{itemize}
		\item
			\texttt{debian/po/*po} files
		\item
			\texttt{debian/po/templates.pot} file
		\item
			\texttt{debian/po/POTFILES.in} file
		\end{itemize}
	\end{block}
	\begin{block}{Translated material}
		\begin{itemize}
		\item
			debian/package/DEBIAN/templates files
		\item
			Never touch it
		\end{itemize}
	\end{block}
\end{frame}

\begin{frame}
  \frametitle{Recommendations}
	\begin{block}
		{Developers Reference}
		\begin{itemize}
		\item
			Follow suggestions in section FIXME to write and translate templates
		\end{itemize}
	\end{block}
	\begin{block}
		{Communicate with translators}
		\begin{itemize}
		\item
			Warn before uploading changed templates
		\item
			podebconf-report-po
		\end{itemize}
	\end{block}
	\begin{block}
		{Provide up-to-date POT files}
		\begin{itemize}
		\item
			debconf-updatepo in clean target
		\end{itemize}
	\end{block}
\end{frame}
% end Christian

\subsubsection{Packages descriptions (DDTP)}
% begin Javier
%     --> focus on "could be the starter for 
%         general i18n infrastructure"

\begin{frame}
  \frametitle{Javier}
\end{frame}

\begin{frame}
  \frametitle{Javier}
\end{frame}

\begin{frame}
  \frametitle{Javier}
\end{frame}
% end Javier

% begin Christian
\subsection{Translation of installed systems}

\begin{frame}
  \frametitle{Goals}
	\begin{block}
		{Get the system as fully localized as possible}
	\end{block}
	\begin{block}
		{As few manual actions as possible for users}
	\end{block}
	\begin{block}
		{Focus on desktop systems}
	\end{block}
\end{frame}

\begin{frame}
  \frametitle{Localization-config}
	\begin{block}
		{Came from D-I/Debian-Edu needs}
		\begin{itemize}
		\item
			Localize gdm, KDE, etc.
		\end{itemize}
	\end{block}
	\begin{block}
		{System-wide tool}
		\begin{itemize}
		\item
			Localisation of user environment is language-env job
		\end{itemize}
	\end{block}
	\begin{block}
		{Should by obsoleted slowly by a general localisation policy}
	\end{block}
\end{frame}

\begin{frame}
  \frametitle{Language tasks}
	\begin{block}
		{Part of tasksel}
	\end{block}
	\begin{block}
		{\texttt{language task}}
		\begin{itemize}
		\item
			Depend only on standard packages
		\item
			Dictionaries, input methods, console display...
		\end{itemize}
	\end{block}
	\begin{block}
		{\texttt{language-desktop task}}
		\begin{itemize}
		\item
			Related to the desktop task
		\item
			Localisation for all software part of the desktop task
		\end{itemize}
	\end{block}
	\begin{block}
		{Most language tasks are not enough maintained}
	\end{block}
\end{frame}
% end Christian

% begin Javier
\subsection{Translation of documentation}

\begin{frame}
  \frametitle{Javier}
\end{frame}

\begin{frame}
  \frametitle{Javier}
\end{frame}

\begin{frame}
  \frametitle{Javier}
\end{frame}

\begin{frame}
  \frametitle{Javier}
\end{frame}

\begin{frame}
  \frametitle{Javier}
\end{frame}
% end Javier

\section{i18n/l10n infrastructure}

% begin Christian
\subsection{PO translation statistics}

\begin{frame}
  \frametitle{PO-debconf, PO statistics web pages}
	\begin{block}
		{The oldest ``framework'' for localisation}
	\end{block}
	\begin{block}
		{Not only statistics, but access to material}
	\end{block}
	\begin{block}
		{Material sorted by popcon scores}
	\end{block}
	\begin{block}
		{Some weaknesses?}
		\begin{itemize}
		\item
			Run under developers accounts
		\item
			Only reflect uploaded packages
		\item
			No status for testing
		\end{itemize}
	\end{block}
\end{frame}

% Here better show one or two pages from the stats pages
% \begin{frame}
%  \frametitle{}
% \end{frame}

% \begin{frame}
%   \frametitle{}
% \end{frame}
% end Christian

% begin Javier
\subsection{Website translation statistics}

\begin{frame}
  \frametitle{Javier}
\end{frame}

\begin{frame}
  \frametitle{Javier}
\end{frame}

\begin{frame}
  \frametitle{Javier}
\end{frame}
% end Javier

% begin Christian
\subsection{Installer translation statistics}

\begin{frame}
  \frametitle{Key points}
	\begin{block}
		{A translated installer is the entry point to a localised system}
	\end{block}
	\begin{block}
		{New languages come through D-I}
	\end{block}
	\begin{block}
		{Work as close of development code as possible}
	\end{block}
\end{frame}

\begin{frame}
  \frametitle{Stats pages and the levels}
	\begin{block}
		{A all in one page}
	\end{block}
	\begin{block}
		{Direct access to material, grabbed from development trees}
	\end{block}
	\begin{block}
		{Weaknesses}
		\begin{itemize}
		\item
			Depend on seppy
		\item
			Sensitive to upstream reorganisations/moves
		\item
			Does not scale well
		\item
			Need a coordinator
		\end{itemize}
	\end{block}
	\begin{block}
		{Strengths}
		\begin{itemize}
		\item
			Big reactivity
		\item
			Enforced ``localisation helpers''
		\end{itemize}
	\end{block}
\end{frame}

\begin{frame}
  \frametitle{}
\end{frame}
% end Christian

% begin Javier
\subsection{Translation robots}

\begin{frame}
  \frametitle{}
\end{frame}

\begin{frame}
  \frametitle{}
\end{frame}

\begin{frame}
  \frametitle{}
\end{frame}

\begin{frame}
  \frametitle{}
\end{frame}
% end Javier

\section{i18n/l10n tools in Debian}

% begin Christian
\subsection{Generic tools}

\begin{frame}
  \frametitle{Gettext as translation common ground}
	\begin{block}
		{Gettext in software}
		\begin{itemize}
		\item
			De facto standard for i18n of Free Software since 1994
		\item
			\texttt{xgettext} to extract strings from source code
		\end{itemize}
	\end{block}
	\begin{block}
		{Gettext for translators}
		\begin{itemize}
		\item
			Begin from POT files
		\item
			Maintain PO files
		\end{itemize}
	\end{block}
	\begin{block}
		{Gettext for both}
		\begin{itemize}
		\item
			\texttt{aptitude install gettext}
		\item
			\texttt{dpkg -L gettext | grep "/bin/msg"}
		\end{itemize}
	\end{block}
\end{frame}

\begin{frame}
  \frametitle{Working on gettext files}
	\begin{block}
		{Standard editors with gettext handling}
		\begin{itemize}
		\item
			vim gettext mode
		\item
			Emacs PO-mode
		\item
			Limited use because of limited dedicated features
		\end{itemize}
	\end{block}
	\begin{block}
		{Dedicated tools}
		\begin{itemize}
		\item
			Easy view of original, translated strings
		\item
			Easy access to comments
		\item
			Spellchecking and error checking (variable matching)
		\item
			Translation memory
		\item
			KBabel, POedit, gtranslator...
		\end{itemize}
	\end{block}
\end{frame}
% end Christian

% begin Javier
\subsection{Translation headers}

\begin{frame}
  \frametitle{}
\end{frame}

\begin{frame}
  \frametitle{}
\end{frame}

\begin{frame}
  \frametitle{}
\end{frame}
% end Javier

% begin Christian
\subsection{Po4a}

\begin{frame}
  \frametitle{PO for Anything}
	\begin{block}
		{Bring gettext translations to new areas}
		\begin{itemize}
		\item
			Man pages
		\item
			Documentation
		\end{itemize}
	\end{block}
	\begin{block}
		{PO4a in Debian}
		\begin{itemize}
		\item
			Man pages for native packages
		\item
			Experimental status pages for manpages
		\item
			Documentation currently prefers POXML
		\end{itemize}
	\end{block}
\end{frame}

\begin{frame}
  \frametitle{PO4a key utilities}
	\begin{block}
		{po4a-gettextize}
		\begin{itemize}
		\item
			Extract translatable strings
		\item
			Write a POT
		\item
			Extract strings from existing translations
		\item
			Mark everything fuzzy
		\item
			No more translations...
		\end{itemize}
	\end{block}
	\begin{block}
		{po4a-translate}
		\begin{itemize}
		\item
			Gathers original document and translations
		\item
			Produces translated documents
		\item
			Results contain English for fuzzy or untranslated material
		\end{itemize}
	\end{block}
\end{frame}

\begin{frame}
  \frametitle{Good practices with PO4a in packages}
	\begin{block}
		{Distribute PO and POT files with the source}
	\end{block}
	\begin{block}
		{Update them in the \texttt{clean} target}
	\end{block}
	\begin{block}
		{Adopt the recommended file organization}
	\end{block}
\end{frame}
% end Christian

% begin Christian
\subsection{Po-debconf}

\begin{frame}
  \frametitle{Po-debconf tools}
	\begin{block}
		{debconf-updatepo}
		\begin{itemize}
		\item
			Update files in \texttt{debian/po}
		\item
			Needs \texttt{debian/po/POTFILES.in}
		\item
			Should be run before shipping the package (clean target?)
		\end{itemize}
	\end{block}
	\begin{block}
		{po2debconf}
		\begin{itemize}
		\item
			Rebuild templates from \texttt{debian/templates} and PO files
		\item
			Resulting templates file is UTF-8 (defined in \texttt{debian/po/output})
		\end{itemize}
	\end{block}
	\begin{block}
		{podebconf-report-po}
		\begin{itemize}
		\item
			Warn translators for needed updates
		\end{itemize}
	\end{block}
\end{frame}

\begin{frame}
  \frametitle{Po-debconf: interacting}
	\begin{block}
		{What translators expect from maintainers}
		\begin{itemize}
		\item
			Read the Developer's Reference
		\item
			Do not change translatable material constantly
		\item
			Warn for changes
		\item
			Avoid verbose jargon that noone will ever read
		\end{itemize}
	\end{block}
	\begin{block}
		{What maintainers expect from translators}
		\begin{itemize}
		\item
			(kindly) Point errors in templates and i18n
		\item
			Learn to use the BTS
		\item
			Be reactive to update requests
		\item
			Keep l10n quality
		\end{itemize}
	\end{block}
\end{frame}

% end Christian

% begin Javier
\subsection{Doc-check}

\begin{frame}
  \frametitle{}
\end{frame}

\begin{frame}
  \frametitle{}
\end{frame}

\begin{frame}
  \frametitle{}
\end{frame}
% end Javier

% begin Christian
\subsection{Use of the BTS for l10n work}

\begin{frame}
  \frametitle{Standard translation update}
	\begin{block}
		{Severity}
		\begin{itemize}
		\item
			wishlist
		\end{itemize}
	\end{block}
	\begin{block}
		{Tags}
		\begin{itemize}
		\item
			l10n
		\item
			patch
		\end{itemize}
	\end{block}
	\begin{block}
		{Title}
		\begin{itemize}
		\item
%FIXME			\[l10n:\<code\>\] \<type\> translation to \<language\>
		\end{itemize}
	\end{block}
	\begin{block}
		{Material}
		\begin{itemize}
		\item
			Attached to the bug report, optionnally compressed
		\end{itemize}
	\end{block}
\end{frame}

\begin{frame}
  \frametitle{}
\end{frame}
% end Christian

\section{Relationship with other projects}

% begin Javier
\subsection{Translation packaging}

\begin{frame}
  \frametitle{}
\end{frame}
% end Javier

% begin Christian
\subsection{Handling BR for upstream l10n}

\begin{frame}
  \frametitle{}
\end{frame}
% end Christian

% begin Christian
\subsection{Handling errors in upstream translations}

\begin{frame}
  \frametitle{}
\end{frame}
% end Christian

\section{Round table}

\begin{frame}
  \frametitle{Kenshi Muto}
\end{frame}

\begin{frame}
  \frametitle{Jacobo Tarrio}
\end{frame}


\end{document}


